\documentclass[a4paper, 12pt]{article}
\usepackage[a4paper,top=1.5cm, bottom=1.5cm, left=1cm, right=1cm]{geometry}
\usepackage{cmap}					
\usepackage{mathtext} 				
\usepackage[T2A]{fontenc}			
\usepackage[utf8]{inputenc}			
\usepackage[english,russian]{babel}
\usepackage{multirow}
\usepackage{graphicx}
\graphicspath{ {./images/} }
\usepackage{wrapfig}
\usepackage{tabularx}
\usepackage{float}
\usepackage{longtable}
\usepackage{hyperref}
\hypersetup{colorlinks=true,urlcolor=blue}
\usepackage[rgb]{xcolor}
\usepackage{amsmath,amsfonts,amssymb,amsthm,mathtools} 
\usepackage{icomma} 
\usepackage{euscript}
\usepackage{mathrsfs}
\usepackage{enumerate}
\usepackage{caption}
\usepackage{enumerate}
\usepackage{graphicx}
\usepackage{caption}
\usepackage{subcaption}
\usepackage[europeanresistors, americaninductors]{circuitikz}
\DeclareMathOperator{\sgn}{\mathop{sgn}}
\newcommand*{\hm}[1]{#1\nobreak\discretionary{}
	{\hbox{$\mathsurround=0pt #1$}}{}}
\title{\textbf{Измерение иннтенсивности радиационного фона (1.1.4)}}
\author{Моргулёв Илья}
\date{Сентябрь 24, 2023}
\begin{document}
	
	\maketitle
	
	\section{Введение}
	
	\textbf{Цель работы:} применение методов обработки экспериментальных данных для изучения статистических закономерностей при измерении интенсивности радиационного фона.
	\bigskip\\
	\textbf{Оборудование:} счетчик Гейгера-Мюллера (CTC-6), блок питания, компьютер с интерфейсом связи со счетчиком.
	
	\section{Теоретические сведения}
	
	В данной работе измеряется число частиц, проходящих через счетчик за 10 секунд, с помощью которого мы можем найти и количество за 40 секунд. Такие интервалы времени выбраны для того, чтобы продемонстрировать, что при большем времени лучше выполняется нормальное распределение измеряемых величин и гистограмма более симметрична, чем при малых временах, когда при оработке лучше воспользоваться законом Пуассона.
	
	Если случайные события, такие как регистрация частицы счётчиком, однородны во времени и являются независимыми, то результаты их измерений подчиняются распределению Пуассона. Теория вероятности гласит, что в таком случае среднеквадратичная ошибка числа отсчётов, измеренного за некоторый интервал времени, равна квадратному корню из среднего числа отсчётов за тот же интервал:
	
	\begin{equation}
		\sigma = \sqrt{n_0}
	\end{equation}
	 
	
	При проведении многочисленных опытов за $n_0$ принимается среднее арифметическое всех результатов $\overline n$, а стандартная ошибка отклонения $\overline n$ от $n_0$ может быть вычислена по формуле:
	\[ \sigma_{\overline n} = \frac{1}{N} \sqrt{\sum_{i=1}^N(n_i - \overline n)^2}, \] где $N$ - количество измерений, $n_i$ - результат $i$-того измерения. Относительная же погрешность составит: \[\varepsilon_{\overline{n}} = \frac{\sigma_{\overline{n}}}{\overline{n}} = \frac{\sigma}{\overline{n} \sqrt{N}} \approx \frac{1}{\sqrt{\overline{n}N}} \] 
	
	\begin{center}
		\textbf{Распределение Пуассона.}
	\end{center}
	Рассмотрим счетчик, регистрирующий частицы. Найдем вероятность того, что при плостности излучения $\nu$ счетчик сработает $n$ раз за время измерения. Для простоты будем считать, что счетчик обладает единичной площадью.
	
	Представим себе большое число совершенно одинаковых одновременно работающих счетчиков. Обозначим полное число счетчиков буквой $N$. Через них в секунду в среднем проходит $N\nu$ частиц, а за $dt$ пройдет $N\nu dt$ частиц. Если $dt$ достаточно мало, то за это время ни через один счетчик не пройдет двух частиц. Число счетчиков, через которые прошла частица равно $N\nu dt$, а их доля по отношению к общему числу счетчиков: $N\nu dt/N = \nu dt$. Вероятность того, что за время $dt$ через счетчик пройдет частица, равна $\nu dt$.
	
	Вычислим теперь вероятность $P_0(t)$ того, что за время $t$ через счетчик не пройдет ни одной частицы. Количество таких счетчиков в момент $t$ составляет $NP_0 (t)$, а в момент времени $t+dt$ равно $NP_0 (t+dt)$. Это число меньше, чем $NP_0 (t)$, потому что за время $dt$ их число убавится на $NP_0 (t)\nu dt$. Поэтому: \[NP_0 (t+dt) = NP_0(t) - NP_0 (t)\nu dt,\] \[P_0 (t+dt) = P_0(t) - P_0 (t)\nu dt.\] Разделим это равенство на $dt$ и переходя к пределу, получим \[ \frac{dP_0}{dt} = -\nu P_0 \]  Интегрируя, найдем: 
	
	\begin{equation}
		P_0 (t) = e^{-\nu t}
		\label{puas}
	\end{equation}
	
	Вычислим теперь $P_n (t+dt)$ -- вероятность того, что за время $t+dt$ через счетчик пройдет ровно $n$ частиц. число таких счетчиков $NP_n(t+dt)$ состоит из двух частей. Первая часть -- счетчики через которые все частицы прошли за $t$ -- $NP_n(t)(1-\nu dt)$, а вторая -- счетчики, через которые за время $t$ прошло $n-1$ частиц, а последня последняя за время $dt$, их число: $NP_{n-1}(t)\nu dt$. Имеем, следовательно: \[NP_n(t+dt) = NP_n(t)(1-\nu dt) + NP_{n-1}(t)\nu dt.\]
	Разделив на $Ndt$ получаем:\[ \frac{dP_n}{dt} + \nu P_n = \nu P_{n-1}. \]
	Применяя формулу полученную реккурентности, с помощью (\ref{puas}) найдем: \[ P_n = \frac{(\nu t)^n}{n!}e^{-\nu t} \] 
	Заметим теперь, что $\nu t $, которое мы обозначим через $n_0$, равно среднему числу частиц, проходящих через счетчик за время $t$. Формула примет вид:
	
	\begin{equation}
		P_n = \frac{n_0^n}{n!}e^{-\nu t}
	\end{equation}
	
	\section{Ход работы}
	
	\begin{enumerate}
		\item Включаем компьютер и счетчик Гейгера-Мюллера. Начинается основной эксперимент.
		\item Проводим демонстрационный эксперимент. Изучая результаты, мы можем понять, что при увеличении числа измерений:
		\begin{enumerate}
			\item измеряемая величина изменяется случайным образом (флуктуирует);
			\item её среднее значение вначале сильно изменяется, затем выходит на постоянную величину;
			\item погрешность отдельного измерения со временем выходит на постоянную величину;
			\item колебания погрешности среднего значения со временем уменьшаются, сама погрешность тоже уменьшается.
		\end{enumerate}
		\item Проводим основной эксперимент, снимаем результаты, получаем таблицу для количества срабатываний счётчика за $20$ с. В столбиках указаны единицы, в строках десятки. Например, 7 столбик третья строчка --> 37 опыт. На компьютере проведём обработку, аналогичную сделанной в демонстрационном эксперименте. Результаты заносим в табоицы $1$ и $2$ .
		
		\begin{center}
                Таблица 1
                \textbf{Число срабатываний счётчика за 20 с}
			\begin{longtable}{|c|c|c|c|c|c|c|c|c|c|c|}
				\hline
				№ опыта & 1 & 2 & 3 & 4 & 5 & 6 & 7 & 8 & 9 & 10\\
				\hline
                \hline
				0 &     22& 27&	24&	15&	21&	23&	26&	27&	20&	23\\
				10& 	18&	21&	31&	26&	24&	28&	21&	31&	28&	19\\
				20& 	22& 19& 28&	28& 18&	22&	20&	22&	22&	29\\
				30& 	37&	35&	25&	20&	32&	28&	31&	30&	29&	19\\
                \hline
				40& 	21&	28&	25&	25&	26&	18&	27&	19&	21&	20\\
				50&  	18&	28&	24&	15&	24&	30&	32&	27&	25&	22\\
				60&	    35&	28&	18&	31&	31&	30&	21&	19&	21&	22\\
				70&	    27&	20&	22&	25&	27&	16&	31&	27&	20&	24\\
                \hline
				80&	    29&	26&	23&	23&	21&	17&	18&	19&	30&	21\\
				90&     21&	28&	27&	15&	19&	24&	18&	25&	32&	33\\
				100&    27&	21&	29&	25&	26&	28&	27&	25&	22&	32\\
				110&	21&	31&	32&	24&	25&	35&	26&	19&	22&	20\\
                \hline
				120&	28&	24&	31&	24&	15&	21&	14&	28&	20&	29\\
				130&	34&	34&	19&	23&	29&	23&	27&	26&	29&	31\\
				140&	29&	22&	26&	30&	29&	27&	22&	24&	23&	25\\
				150&	22&	30&	32&	26&	16&	29&	29&	24&	22&	21\\
                \hline
				160&	24&	19&	29&	30&	28&	25&	22&	15&	26&	24\\
				170&	24&	17&	26&	12&	24& 16&	26&	16&	16&	24\\
				180&	24&	26&	18&	27&	23&	19&	31&	25&	25&	32\\
				190&	27&	29&	21&	24&	25&	30&	28&	18&	21&	22\\
                \hline
			\end{longtable}
		\end{center}
		\item Переносим также данные для $\tau = 10\: с$ и строим по ним гистограмму распределения числа отсчётов $\omega_n = f(n)$:
		\begin{center}
                Таблица 2
                \textbf{Данные для построения гистограммы распределения числа срабатываний счётчика за 10 с}
			\begin{longtable}{|c|c|c|c|c|c|c|c|c|c|c|c|c|}
				\hline
				Число импульсов & 3 & 4 & 5 & 6 & 7 & 8 & 9 \\
				\hline
				Число случаев & 0 & 0 & 2 & 17 & 17 & 23 & 26\\
				\hline
				Доля случаев & 0 & 0 & 0.005 & 0.0425 & 0.0425 & 0.0575 & 0.065\\
				\hline
				\hline
				Число импульсов & 10 & 11 & 12 & 13 & 14 & 15 & 16\\
				\hline
				Число случаев & 36 & 44 & 51 & 44 & 46 & 30 & 19\\
				\hline
				Доля случаев & 0.09 & 0.11 & 0.1275 & 0.11 & 0.115 & 0.075 & 0.0475 \\
				\hline
				\hline
				Число импульсов & 17 & 18 & 19 & 20 & 21 & 22 & 23\\
				\hline
				Число случаев & 17 & 11 & 6 & 8 & 1 & 1 & 1\\ 
				\hline
				Доля случаев & 0.0425 & 0.0275 & 0.015 & 0.02 & 0.0025 & 0.0025 & 0.0025\\
                \hline
			\end{longtable}
		\end{center}
		
		\item Разобьём полученные результаты для $\tau = 20 \: c$ на группы по два в порядке их следования и построим гистограмму для $\tau = 40 \: c$. На гистограмме отобразим случай, долю случая и апроксимацию по среднему скользящему значению.
		\begin{center}
                Таблица 3
                \textbf{Число срабатываний счётчика за 40 с}
			\begin{tabular}{|c|c|c|c|c|c|c|c|c|c|c|}
				\hline
				№ опыта & 1 & 2 & 3 & 4 & 5 & 6 & 7 & 8 & 9 & 10\\ \hline
				0 & 49 & 39 & 44 & 53 & 43 &  39 & 57 & 52 & 52 & 47\\ \hline 
				10 & 41 & 56 & 40 & 42 & 51 & 72 & 45 & 60 & 61 & 48\\ \hline 
				20 & 49 & 50 & 44 & 46 & 41 & 46 & 39 & 54 & 59 & 47\\ \hline 
				30 & 63 & 49 & 61 & 40 & 43 & 47 & 47 & 43 & 58 & 44\\ \hline 
				40 & 55 & 46 & 38 & 37 & 51 & 49 & 42 & 43 & 43 & 65\\ \hline 
				50 & 48 & 54 & 54 & 52 & 54 & 52 & 56 & 60 & 45 & 42\\ \hline 
				60 & 52 & 55 & 36 & 42 & 49 & 68 & 42 & 52 & 53 & 60\\ \hline 
				70 & 51 & 56 & 56 & 46 & 48 & 52 & 58 & 45 & 53 & 43\\ \hline 
				80 & 43 & 59 & 53 & 37 & 50 & 41 & 38 & 40 & 42 & 40\\ \hline 
				90 & 50 & 45 & 42 & 56 & 57 & 56 & 45 & 55 & 46 & 43\\ \hline 
			\end{tabular}
		\end{center}
		Данные для $\tau = 40\ c$:
		\begin{center}
                Таблица 4
                \textbf{Данные для построения гистограммы распределения числа срабатываний счётчика за 40 с}
			\begin{tabular}{|c|c|c|c|c|c|c|c|}
				\hline
				Число импульсов & 36 & 37 & 38 & 39 & 40 & 41 & 42\\
				\hline
				Число случаев & 1 & 2 & 2 & 3 & 4 & 3 & 5\\
				\hline
				Доля случаев & 0.0105 & 0.021 & 0.021 & 0.0315 & 0.042 & 0.0315 & 0.0525\\
				\hline \cline{1-8}
				Число импульсов & 43 & 44 & 45  & 46 & 47 & 48 & 49\\
				\hline
				Число случаев & 8 & 3 & 5 & 5 & 4 & 3 & 4\\
				\hline
				Доля случаев & 0.084 & 0.0315 & 0.0525  & 0.0525 & 0.042 & 0.0315 & 0.042\\
				\hline \cline{1-8}
				Число импульсов & 50 & 51 & 52 & 53 & 54 & 55 & 56\\
				\hline
				Число случаев & 3 & 3 & 7 & 4 & 4 & 3 & 6\\
				\hline
				Доля случаев & 0.0315 & 0.0315 & 0.0735 & 0.042 & 0.042 & 0.0315 & 0.063\\
				\hline \cline{1-8}
				Число импульсов & 57 & 58 &  59 & 60 & 61 & 65 & 68\\
				\hline
				Число случаев & 2 & 2 & 2 & 3 & 2 & 1 & 1\\
				\hline
				Доля случаев & 0.021 & 0.021 & 0.021 & 0.0315 & 0.021 & 0.0105 & 0.0105\\
				\hline \cline{1-8} \cline{1-3}
			\end{tabular}
		\end{center}
		\begin{figure}[h!]
			\centering
			\includegraphics[scale = 0.56]{chart (3) (1).png}
			\caption{Гисторамма, полученная путём наложения гистограмм для $\tau = 10\: c $ и $\tau = 40\ с$ для $\omega_n (n)$}
			\label{fig:my_label}
		\end{figure}
		\item Для обоих измерений определим среднее число частиц $\overline{n}$, среднеквадратичное отклонение отдельного измерения $\sigma_{отд}$ и среднего значения $\sigma_{\overline n}$:
		\[ \overline{n} = \frac{1}{N}\sum_{i=1}^N n_i, \:\:\:\:\: \sigma_{} = \sqrt{\frac{1}{N} \sum_{i=1}^{N}(n_i - \overline{n})^2}, \:\:\:\:\: \sigma_{\overline n} = \frac{1}{N} \sqrt{\sum_{i=1}^N(n_i - \overline n)^2} = \frac {\sigma_{}}{\sqrt{N}}, \] а также убедимся в справедливости формулы Пуассона $\sigma_{} \approx \sqrt{\overline n}$.
		\par
		$t = 10 \: c$: $\overline n_1 = 12.26, \:\sigma_{1} = 3.5 ,\: \sigma_{\overline n_1} = 0,175$; $3.5 \approx \sqrt{12.26} = 3.501$.
		\par
		$t = 40 \: c$: $\overline n_2 = 49,3, \:\sigma_{2} = 7,16, \: \sigma_{\overline n_2} = 0,716$; $7,16 \approx \sqrt{49,3} = 7.02$.
            \item Сравним среднеквадратичные ошибки отдельных измерений для двух распределений легко увидеть, что хотя абсолютное значение $\sigma$ во втором распределнии больше, чем в первом, относительная полуширина второго распределения меньше:
                \[\frac{\sigma_{1}}{\overline{n_{1}}} \approx 0.29
                \:\:\:\: \frac{\sigma_{2}}{\overline{n_{2}}} \approx 0.15\]
            это так же следует из рис. 2.
            \item найдём относительную ошибку $\varepsilon$ для $\overline{n}_1$ и $\overline{n}_2$ :
            \[ \varepsilon_{\overline{n}_1} = \frac{\sigma_{\overline{n}_1}}{\overline{n}_1} \cdot 100\% = \frac{0.175}{12.26} \cdot 100\% \approx 1.42\% \:\:\:\: \varepsilon_{\overline{n}_2} = \frac{\sigma_{\overline{n}_2}}{\overline{n}_2} \cdot 100\% = \frac{0.716}{49.3} \cdot 100\% \approx 1.45\%
            \]
            используя формулу (7) $\varepsilon_{\overline{n}} = \frac{\sigma_{\overline{n}}}{\overline{n}} = \frac{\sigma}{\overline{n} \sqrt{N}} \approx \frac{1}{\sqrt{\overline{n}N}}$ :
            \[ \varepsilon_{\overline{n}_2}  = \frac{100\%}{\sqrt{\overline{n}_2N_2}} = \frac{100\%}{\sqrt{49.3 \cdot 100}} \approx 1.42\% \approx \varepsilon_{\overline{n}_1}
            \]
            \item из п. 6 получаем окончательные результаты для $\tau = 10\ c$ и $\tau = 40\ c$:
            \[ n_{t=10c} = \overline{n}_1 \pm \sigma_{\overline{n}_1} = 12.26 \pm 0.175
            \]
            \[ n_{t=40} = \overline{n}_2 \pm \sigma_{\overline{n}_2} = 49.3 \pm 0.716
            \]
		\item Найдём долю случаев, когда отклонение от среднего не превышает $\sigma, 2\sigma$. Сравним результаты с теоретическими оценками.
		\begin{center}
                Таблица 5
			\begin{tabular}{|c|c|c|}
				\hline
				Ошибка & Доля случаев, \% & Теоретическая оценка \\\hline
				$\pm\sigma_1 = \pm3.5$ & 68.5 & 68\\\hline
				$\pm2\sigma_1 = \pm7$ & 95.2 & 95\\\hline \hline
				$\pm\sigma_2 = \pm7.16$ & 66.5 & 68\\\hline
				$\pm2\sigma_2 = \pm14.32$ & 96,8 & 95\\\hline 
			\end{tabular}    
		\end{center}
	\end{enumerate}
\section{Обсуждение результатов}
В ходе работы была произведена обработка данных в двух сериях экспериметов: с временем эксперимета 10 с и с временем эксперимета 40 с. Получены результаты соответственно $\overline n_1 = 12,26 \pm 0,175$ и   $\overline n_2 = 49,3 \pm 0,716$. Относительные ошибки определения $n_1$ и $n_2$ совпадают и весьма невелики ($1,42 \%$). Проверено в п. 6, что результаты измерений соответствуют характерному для распределения Пуассона равенству: $\sigma = \sqrt{n_0}$.
\end{document}
